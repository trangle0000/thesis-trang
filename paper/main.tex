% Style for a MSc paper at Warsaw School of Economics
% Michał Ramsza
% Fri Dec 22 14:49:36 CET 2023

% --- document class and other global stuff ---------------------------
\documentclass[english, twoside, 12pt, a4paper]{article}

% --- packages --------------------------------------------------------
\usepackage{textcomp}
\usepackage{times}
\usepackage{amsmath}
\usepackage{amsfonts}
\usepackage{amssymb}
\usepackage{amsthm}
\usepackage[T1]{fontenc}
\usepackage[utf8]{inputenc}
\usepackage{graphicx}
\usepackage{tikz}
\usepackage{xcolor}
\usepackage{enumitem}
\usepackage[english]{babel}
\usepackage{booktabs}
\usepackage{setspace}
\usepackage[centering, left=3.5cm, right=2.5cm, textheight=24cm]{geometry}

% --- packages for citations ------------------------------------------
\usepackage[backend=biber, style=authoryear, autocite=inline, defernumbers=true]{biblatex}
\addbibresource{refs.bib}

% --- package for automatic insertion of R code -----------------------
\usepackage{listings}
\lstset{%
   numbers=left,%
   tabsize=3,%
   numberstyle=\footnotesize,%
   basicstyle=\ttfamily \small \color{black},%
   keywordstyle=\ttfamily \small \color{black},%
   commentstyle=\ttfamily \small \color{gray},%
   stringstyle=\ttfamily \small \color{black},%
   identifierstyle=,%
   showstringspaces=false,%
   escapeinside={(*@}{@*)}}   
   
\lstset{
 literate={ą}{{\k a}}1
 {Ą}{{\k A}}1
 {ż}{{\. z}}1
 {Ż}{{\. Z}}1
 {ź}{{\' z}}1
 {Ź}{{\' Z}}1
 {ć}{{\' c}}1
 {Ć}{{\' C}}1
 {ę}{{\k e}}1
 {Ę}{{\k E}}1
 {ó}{{\' o}}1
 {Ó}{{\' O}}1
 {ń}{{\' n}}1
 {Ń}{{\' N}}1
 {ś}{{\' s}}1
 {Ś}{{\' S}}1
 {ł}{{\l}}1
 {Ł}{{\L}}1
}   

% --- support for links -----------------------------------------------	
\usepackage{hyperref}
\hypersetup{colorlinks=true,
            linkcolor=black,
            citecolor=darkgray,
            urlcolor=darkgray}
\usepackage{xurl}            
\urlstyle{same}

% --- support for large tables and other stuff ------------------------	
\usepackage{float}
\usepackage{caption}
\usepackage{subcaption}
\usepackage{wrapfig}

% --- support for game theory ------------------------------------------
\usepackage{sgame}

% --- support for no widows --------------------------------------------
\usepackage[defaultlines=4,all]{nowidow}

% -----------------------------------------------------------
\usepackage{setspace}

% --- definitions for environments -------------------------------------
\theoremstyle{definition}
    \newtheorem{condition}{Assumption}
    \newtheorem{example}{Example}      

\theoremstyle{plain}
    \newtheorem{definition}{Definition}    
    \newtheorem{proposition}{Proposition}
    \newtheorem{theorem}{Theorem}
    \newtheorem{cor}{Corollary}

\theoremstyle{remark}
    \newtheorem{remark}{Remark}

% --- other settings --------------------------------------------------
\linespread{1.5}
\frenchspacing
\sloppy
\allowdisplaybreaks[4]
\raggedbottom
\clubpenalty=10000
\widowpenalty=10000

% --- only if required ------------------------------------------------
\AtBeginDocument{\renewcommand*{\figurename}{Figure}}
\AtBeginDocument{\renewcommand*{\tablename}{Table}}

% --- changing definition of footnote ---------------------------------
\makeatletter
\renewcommand\footnotesize{%
   \@setfontsize\footnotesize\@ixpt{10}%
   \abovedisplayskip 8\p@ \@plus2\p@ \@minus4\p@
   \abovedisplayshortskip \z@ \@plus\p@
   \belowdisplayshortskip 4\p@ \@plus2\p@ \@minus2\p@
   \def\@listi{\leftmargin\leftmargini
               \topsep 4\p@ \@plus2\p@ \@minus2\p@
               \parsep 2\p@ \@plus\p@ \@minus\p@
               \itemsep \parsep}%
   \belowdisplayskip \abovedisplayskip
}
\makeatother

% --- useful definitions ----------------------------------------------
\newcommand{\code}[1]{\lstinline{#1}}

% ---------------------------------------------------------------------
\begin{document}

% --- strona tytulowa -------------------------------------------------
\begin{titlepage}
\centering

\includegraphics[width=0.66\textwidth]{logo.JPG}

\vspace*{0.5cm}
Studium <licencjackie/magisterskie>\\
\begin{flushleft}
Kierunek: Metody Ilościowe w Ekonomii i System Informacyjne\\
%Specjalność: <specjalność> % w przypadku braku należy pominać
%Forma studiów: <forma studiów (stacjonarne, itd.)>
\end{flushleft}

\vspace*{.5cm}
\rule{0cm}{1cm}\hfill
\begin{minipage}{9cm}
Author's name and family name: Michał Ramsza\\
Nr albumu: <12345>
\end{minipage}

\vspace*{1cm}
\begin{minipage}{12cm}
\centering
\Large
\textbf{<title>}
\end{minipage}

\vspace*{2cm}
\rule{0cm}{1cm}\hfill
\begin{minipage}{9cm}
Praca <licencjacka/magisterska> napisana\\
w Instytucie Ekonomii Matematycznej\\
pod kierunkiem naukowym\\
dr hab. Michała Ramszy
\end{minipage}

\vfill
Warszawa <year>
\end{titlepage}

\rule{1ex}{0ex}\clearpage

% --- table of contents -----------------------------------------------
\cleardoublepage
\tableofcontents

% --- chapter ---------------------------------------------------------
\cleardoublepage
\section{Introduction}



\section{Conclusions}

% --- bibliography ----------------------------------------------------
\clearpage
\printbibliography[heading=subbibliography,nottype=online,title={References}]
\printbibliography[heading=subbibliography,type=online,title={Online references}]


% --- abstract --------------------------------------------------------
\clearpage
\addcontentsline{toc}{section}{List of tables}
\listoftables

% --- abstract --------------------------------------------------------
\clearpage
\addcontentsline{toc}{section}{List of figures}
\listoffigures



% --- abstract --------------------------------------------------------
\clearpage
\addcontentsline{toc}{section}{Streszczenie}
\section*{Streszczenie}

Tutaj zamieszczają Państwo streszczenie pracy. Streszczenie powinno być długości około pół strony.


\end{document}


%%% Local Variables:
%%% mode: latex
%%% TeX-master: t
%%% End:
