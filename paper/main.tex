% Style for a MSc paper at Warsaw School of Economics
% Michał Ramsza
% Fri Dec 22 14:49:36 CET 2023

% --- document class and other global stuff ---------------------------
\documentclass[english, twoside, 12pt, a4paper]{article}

% --- packages --------------------------------------------------------
\usepackage{textcomp}
\usepackage{times}
\usepackage{amsmath}
\usepackage{amsfonts}
\usepackage{amssymb}
\usepackage{amsthm}
\usepackage[T1]{fontenc}
\usepackage[utf8]{inputenc}
\usepackage{graphicx}
\usepackage{tikz}
\usepackage{xcolor}
\usepackage{enumitem}
\usepackage[english]{babel}
\usepackage{booktabs}
\usepackage{setspace}
\usepackage[centering, left=3.5cm, right=2.5cm, textheight=24cm]{geometry}

% --- packages for citations ------------------------------------------
\usepackage[backend=biber, style=authoryear, autocite=inline, defernumbers=true]{biblatex}
\addbibresource{refs.bib}

% --- package for automatic insertion of R code -----------------------
\usepackage{listings}
\lstset{%
   numbers=left,%
   tabsize=3,%
   numberstyle=\footnotesize,%
   basicstyle=\ttfamily \small \color{black},%
   keywordstyle=\ttfamily \small \color{black},%
   commentstyle=\ttfamily \small \color{gray},%
   stringstyle=\ttfamily \small \color{black},%
   identifierstyle=,%
   showstringspaces=false,%
   escapeinside={(*@}{@*)}}   
   
\lstset{
 literate={ą}{{\k a}}1
 {Ą}{{\k A}}1
 {ż}{{\. z}}1
 {Ż}{{\. Z}}1
 {ź}{{\' z}}1
 {Ź}{{\' Z}}1
 {ć}{{\' c}}1
 {Ć}{{\' C}}1
 {ę}{{\k e}}1
 {Ę}{{\k E}}1
 {ó}{{\' o}}1
 {Ó}{{\' O}}1
 {ń}{{\' n}}1
 {Ń}{{\' N}}1
 {ś}{{\' s}}1
 {Ś}{{\' S}}1
 {ł}{{\l}}1
 {Ł}{{\L}}1
}   

% --- support for links -----------------------------------------------	
\usepackage{hyperref}
\hypersetup{colorlinks=true,
            linkcolor=black,
            citecolor=darkgray,
            urlcolor=darkgray}
\usepackage{xurl}            
\urlstyle{same}

% --- support for large tables and other stuff ------------------------	
\usepackage{float}
\usepackage{caption}
\usepackage{subcaption}
\usepackage{wrapfig}

% --- support for game theory ------------------------------------------
\usepackage{sgame}

% --- support for no widows --------------------------------------------
\usepackage[defaultlines=4,all]{nowidow}

% -----------------------------------------------------------
\usepackage{setspace}

% --- definitions for environments -------------------------------------
\theoremstyle{definition}
    \newtheorem{condition}{Assumption}
    \newtheorem{example}{Example}      

\theoremstyle{plain}
    \newtheorem{definition}{Definition}    
    \newtheorem{proposition}{Proposition}
    \newtheorem{theorem}{Theorem}
    \newtheorem{cor}{Corollary}

\theoremstyle{remark}
    \newtheorem{remark}{Remark}

% --- other settings --------------------------------------------------
\linespread{1.5}
\frenchspacing
\sloppy
\allowdisplaybreaks[4]
\raggedbottom
\clubpenalty=10000
\widowpenalty=10000

% --- only if required ------------------------------------------------
\AtBeginDocument{\renewcommand*{\figurename}{Figure}}
\AtBeginDocument{\renewcommand*{\tablename}{Table}}

% --- changing definition of footnote ---------------------------------
\makeatletter
\renewcommand\footnotesize{%
   \@setfontsize\footnotesize\@ixpt{10}%
   \abovedisplayskip 8\p@ \@plus2\p@ \@minus4\p@
   \abovedisplayshortskip \z@ \@plus\p@
   \belowdisplayshortskip 4\p@ \@plus2\p@ \@minus2\p@
   \def\@listi{\leftmargin\leftmargini
               \topsep 4\p@ \@plus2\p@ \@minus2\p@
               \parsep 2\p@ \@plus\p@ \@minus\p@
               \itemsep \parsep}%
   \belowdisplayskip \abovedisplayskip
}
\makeatother

% --- useful definitions ----------------------------------------------
\newcommand{\code}[1]{\lstinline{#1}}

% ---------------------------------------------------------------------
\begin{document}

% --- strona tytulowa -------------------------------------------------
\begin{titlepage}
\centering

\includegraphics[width=0.66\textwidth]{logo.JPG}

\vspace*{0.5cm}
Studium <licencjackie/magisterskie>\\
\begin{flushleft}
Kierunek: Metody Ilościowe w Ekonomii i System Informacyjne\\
%Specjalność: <specjalność> % w przypadku braku należy pominać
%Forma studiów: <forma studiów (stacjonarne, itd.)>
\end{flushleft}

\vspace*{.5cm}
\rule{0cm}{1cm}\hfill
\begin{minipage}{9cm}
Author's name and family name: Michał Ramsza\\
Nr albumu: <12345>
\end{minipage}

\vspace*{1cm}
\begin{minipage}{12cm}
\centering
\Large
\textbf{<title>}
\end{minipage}

\vspace*{2cm}
\rule{0cm}{1cm}\hfill
\begin{minipage}{9cm}
Praca <licencjacka/magisterska> napisana\\
w Instytucie Ekonomii Matematycznej\\
pod kierunkiem naukowym\\
dr hab. Michała Ramszy
\end{minipage}

\vfill
Warszawa <year>
\end{titlepage}

\rule{1ex}{0ex}\clearpage

% --- table of contents -----------------------------------------------
\cleardoublepage
\tableofcontents

% --- chapter ---------------------------------------------------------
\cleardoublepage
\section{Introduction}

The pre-owned car market has become an increasingly important segment of the automotive industry, driven by various factors such as economic fluctuations, consumer preferences, and advancements in technology. This market encompasses a diverse range of vehicles, from certified pre-owned cars to older models, each appealing to different buyer demographics. With the rising costs of new cars and the growing awareness of sustainability, consumers are increasingly turning to pre-owned options as a cost-effective and environmentally friendly alternative. Furthermore, the digital transformation has allowed for enhanced accessibility and transparency in the buying process, with online platforms facilitating better comparisons and evaluations of pre-owned vehicles.

This is citation \textcite{vanderAalst2022} and this is\footfullcite{vanderAalst2022}.

The formula for the pdf of the normal distribution is given by \eqref{eq:pdfnormal}. 

This is math inline \(f(x) = x^2 - \sin(x)\), $f(x) = x^2 - \cos(x)$. This is a formula for the cumulative probability function
\begin{equation}
 F(x) = \int\limits_{-\infty}^{x}       
f(        \tau) d\tau                  . 
\end{equation}

This is an equation that is numbered
\begin{equation}\label{eq:pdfnormal}
f(x) = \frac{e^{-\frac{(x-\mu )^2}{2
   \sigma ^2}}}{\sqrt{2 \pi }
   \sigma } 
\end{equation}

The automotive industry is undergoing a fundamental transformation as traditional vehicle ownership models give way to innovative mobility solutions that prioritize access over ownership. Beyond Cars represents a paradigm shift in how consumers interact with transportation, encompassing services such as car-sharing platforms, subscription-based vehicle access, autonomous ride-hailing services, and integrated mobility-as-a-service (MaaS) ecosystems. This emerging market segment challenges conventional automotive business models by offering consumers flexible, on-demand access to vehicles without the financial burden and maintenance responsibilities associated with traditional car ownership. The concept appeals particularly to urban populations, younger demographics, and environmentally conscious consumers who value convenience, cost-effectiveness, and sustainability over the status and control traditionally associated with vehicle ownership.

The Beyond Cars market operates within a complex ecosystem of technological innovation, regulatory frameworks, and changing consumer preferences that collectively drive its rapid evolution. Key market drivers include urbanization trends that make car ownership less practical, advances in digital platforms that enable seamless vehicle sharing and booking systems, and growing environmental awareness that favors shared mobility solutions over individual vehicle ownership. The market structure is characterized by diverse business models ranging from peer-to-peer car sharing platforms to corporate fleet management services, each targeting specific consumer segments and use cases. Understanding this market requires analyzing not only the direct economic impacts on traditional automotive sales but also the broader implications for urban planning, insurance models, and the development of supporting infrastructure such as charging networks for electric vehicles and innovative parking solutions.



% --- chapter ---------------------------------------------------------
\clearpage
\section{Basic things}

\subsection{Compiling \LaTeX files}

The \verb+.tex+ file is just a plain text file. It contains the \LaTeX{} formatting codes together with the content of a paper. To get a \verb+.pdf+ file you have to compile the \verb+.tex+ file using a sequence \verb+pdflatex+, \verb+biblatex+, \verb+pdflatex+, \verb+pdflatex+. This sequence is a default in most editors designed for use with \LaTeX.

\subsection{Basic formatting for a text}

Paragraphs are coded by an empty line. That is is you want to start a new paragraph it is enough to leave an empty line and start typing like that:
\begin{verbatim}
This is the first paragraph.

This is the next paragraph.
\end{verbatim}

Everything about the paragraph is formatted for you including all indents and spacings. Again, you don't have to take care of it manually.

Basic text formatting, e.g. bold face and italic, is achieved with the following commands: \verb+\textbf{}+, \verb+\textit{}+, \verb+\underline{}+, producing \textbf{text}, \textit{text}, \underline{text}. I suggest not overusing those commands!

Alignment is done through environments \verb+center+, \verb+flushleft+ and \verb+\flushright+ giving the following examples.

\begin{center}
  This is centered.
\end{center}

\begin{flushleft}
  This is aligned to the left.
\end{flushleft}

\begin{flushright}
  This is aligned to the right. 
\end{flushright}

In other environments it is possible to use \verb+\centering+ to center content of that environment (like in \verb+figure+ or \verb+table+ environments).

\subsection{Fonts and fonts' sizes}

You do not change fonts and fonts' sizes! Technically it can be done but I will reject this.

% --- chapter ---------------------------------------------------------
\clearpage
\section{Mathematics}

This is testing footnotes\footnote{This is a footnote. We can put some math here \( x^2 - f(x) = g(x^2) \) which is not encouraged but sometimes necessary. The other thing we can do is to put here an URL \url{https://tex.stackexchange.com/questions/249415/set-font-size-for-footnotes}. }.

\subsection{Basic mathematics}

There are two types of mathematics inside a \LaTeX{} document. The first one is the in-line mathematics and the displayed mathematics. The first one looks like this: \( F(x) = \int_{-\infty}^{x} f(\omega) d\omega \) with the code looking like this: \verb!\( F(x) = \int_{-\infty}^{x} f(\omega) d\omega \)!. The displayed mathematics looks like that
\[
F(x) = \int_{-\infty}^{x} f(\omega) d\omega
\]
with the code
\begin{verbatim}
\[
F(x) = \int_{-\infty}^{x} f(\omega) d\omega
\]
\end{verbatim}
As you can see the same code is formatted differently depending on the type of mathematics.

\subsection{Referencing mathematics and other things}

To reference mathematics (only displayed formulas) you use the \verb+equation+ environment with a \verb+\label{}+ within. The reference is done through the \verb+\ref{}+ command. The example is
\begin{equation}
\label{eq:this-is-very-important-equation}
F(x) = \int_{-\infty}^{x} f(\omega) d\omega.
\end{equation}
To reference the equation you use the \verb+\ref{}+ command giving (\ref{eq:this-is-very-important-equation}). The \verb+\label{}+ / \verb+\ref{}+ pair works for anything that can be referenced.

\subsection{Some more mathematical formulas}

\LaTeX{} is known for producing beautifully typeset mathematical formulas. The above mathematical formulas are relatively simple. Here are slightly more complex formulas. Let \(A\) be a matrix
\[
A =
\left(
\begin{bmatrix}
1                   & \alpha^2       \\
2                   & \sqrt{\pi} - \log(x-\sin(y))
\end{bmatrix}^{2}
- 
\begin{bmatrix}
1                   & f(x)           \\
2                   & g(y)
\end{bmatrix}
\cdot
\begin{bmatrix}
x                                    \\
y
\end{bmatrix}
\right),
\]
where
\[
f(x) = 
\left\{
  \begin{aligned}
    \frac{1}{x}     & \quad \text{for \(x<-\frac{1}{2}\),} \\
    \frac{1}{1+x^2} & \quad \text{for \(x \geq -\frac{1}{2}\)}
  \end{aligned}
\right.
\]
and
\[
g(y) = \sin\left(\frac{\mathrm{\mathbf{E}}(X)}{\cos(y) + \log(y)}\right), 
\quad\text{where \( X \sim \mathrm{N}(0, \sigma)  \).}
\]
Note that the above formulas are parts of a sentence. Thus, you still use proper punctuation. In \LaTeX{}, we can also typeset diagrams of arbitrary complexity. However, this requires another language for defining graphical scenes: TikZ (\url{https://tikz.org/}). 

\begin{figure}[hbt]
  \centering

\resizebox{0.6\textwidth}{!}{%
\usetikzlibrary {arrows.meta,automata,positioning,shadows}
\begin{tikzpicture}[->,>={Stealth[round]},shorten >=1pt,auto,node distance=2.8cm,on grid,semithick,
                    every state/.style={fill=red,draw=none,circular drop shadow,text=white}]

  \node[initial,state] (A)                    {$q_a$};
  \node[state]         (B) [above right=of A] {$q_b$};
  \node[state]         (D) [below right=of A] {$q_d$};
  \node[state]         (C) [below right=of B] {$q_c$};
  \node[state]         (E) [below=of D]       {$q_e$};

  \path (A) edge              node {0,1,L} (B)
            edge              node {1,1,R} (C)
        (B) edge [loop above] node {1,1,L} (B)
            edge              node {0,1,L} (C)
        (C) edge              node {0,1,L} (D)
            edge [bend left]  node {1,0,R} (E)
        (D) edge [loop below] node {1,1,R} (D)
            edge              node {0,1,R} (A)
        (E) edge [bend left]  node {1,0,R} (A);

   \node [right=3cm,text width=7cm] at (C)
   { \small
     The current candidate for the busy beaver for five states. It is
     presumed that this Turing machine writes a maximum number of
     $1$'s before halting among all Turing machines with five states
     and the tape alphabet $\{0, 1\}$. Proving this conjecture is an
     open research problem.
   };
\end{tikzpicture}
}

  \captionsetup{margin=10pt,font=small,labelfont=bf,width=.8\textwidth}

  \caption[Example of a diagram]{This is a simple diagram of a Turing machine. With TikZ, we can prepare diagrams of any complexity. \textit{Source:} \url{https://tikz.org/}.}\label{fig:diagram}
\end{figure}

For those of you doing game theory, the TikZ is a great solution for visualizing extensive-form games. For normal-form games, we have a  simpler solution. It is very easy to typeset a normal-form game. Below is an example of such a game. 

\begin{center}

\begin{game}{3}{3}
    & $L$    & $M$    & $H$    \\
$L$ & $16,9$ & $3,13$ & $0,3$  \\
$M$ & $21,1$ & $10,4$ & $-1,0$ \\
$H$ & $9,0$  & $5,-4$ & $-5,-15$
\end{game}
 
\end{center}

% --- chapter ---------------------------------------------------------
\clearpage
\section{Figures and tables}

Both figures and tables use the same ideas. To insert a table, you use the \verb+table+ environment. The following tables are just examples of what can be automatically generated with the R and Python programming languages.

\begin{table}[hbt]
  \centering

  \captionsetup{margin=10pt,font=small,labelfont=bf,width=.8\textwidth}

  \caption[Short name for a table]{This is an example of a table generated in the R programming language. The script generating the table is \code{example.R}.}
  \label{tab:exceptional-table}

  \vspace*{2ex}
  \footnotesize

  
% Table created by stargazer v.5.2.3 by Marek Hlavac, Social Policy Institute. E-mail: marek.hlavac at gmail.com
% Date and time: ptk, gru 22, 2023 - 16:48:02
\begin{tabular}{@{\extracolsep{5pt}} cccc} 
\\[-1.8ex]\hline 
\hline \\[-1.8ex] 
 & Values x & Values y & Class \\ 
\hline \\[-1.8ex] 
1 & -0.12 & 0.73 & Down \\ 
2 & -1.54 & -2 & Up \\ 
3 & -0.64 & -0.36 & Down \\ 
4 & -0.96 & -0.43 & Up \\ 
5 & 0.92 & 1.72 & Down \\ 
\hline \\[-1.8ex] 
\end{tabular} 

  
\end{table}

\begin{table}[hbt]
  \centering

  \captionsetup{margin=10pt,font=small,labelfont=bf,width=.8\textwidth}

  \caption[Short name for a table]{This is another example of a table generated in the R programming language. This table is automatically generated from the linear regression model. }
  \label{tab:lm}

  \vspace*{2ex}
  \footnotesize

  
% Table created by stargazer v.5.2.3 by Marek Hlavac, Social Policy Institute. E-mail: marek.hlavac at gmail.com
% Date and time: ptk, gru 22, 2023 - 16:55:58
\begin{tabular}{@{\extracolsep{5pt}}lc} 
\\[-1.8ex]\hline 
\hline \\[-1.8ex] 
 & \multicolumn{1}{c}{\textit{Dependent variable:}} \\ 
\cline{2-2} 
\\[-1.8ex] & y \\ 
\hline \\[-1.8ex] 
 x & 1.989$^{***}$ \\ 
  & (0.032) \\ 
  & \\ 
 Constant & 1.020$^{***}$ \\ 
  & (0.033) \\ 
  & \\ 
\hline \\[-1.8ex] 
Observations & 1,000 \\ 
R$^{2}$ & 0.796 \\ 
Adjusted R$^{2}$ & 0.796 \\ 
Residual Std. Error & 1.029 (df = 998) \\ 
F Statistic & 3,903.749$^{***}$ (df = 1; 998) \\ 
\hline 
\hline \\[-1.8ex] 
\textit{Note:}  & \multicolumn{1}{r}{$^{*}$p$<$0.1; $^{**}$p$<$0.05; $^{***}$p$<$0.01} \\ 
\end{tabular} 

  
\end{table}

To insert a figure, you need to have a figure. In the \code{./figs} directory, there are figures generated with the R and Python scripts, and the following is an example of the \verb+figure+ environment. Figure~\ref{fig:xxx} is slightly more complex than just a simple figure, but it is useful to have such a template. It is possible to reference subfigures as \ref{fig:xxxa} and \ref{fig:xxxb}.

\begin{figure}[hbt]
  \centering

  \begin{subfigure}[t]{0.45\textwidth}
    \includegraphics[width=\textwidth]{./figs/fig_00.png}
  \end{subfigure}

  \captionsetup{margin=10pt,font=small,labelfont=bf,width=.8\textwidth}

  \caption[Short name]{This is an example figure generated in the R programming language.  \textit{Source:} own calculations.}\label{fig:xxx1}
\end{figure}

\begin{figure}[hbt]
  \centering
  \begin{subfigure}[t]{0.45\textwidth}
    \includegraphics[width=\textwidth]{./figs/fig_01.png}
    \caption{This is another visualization done in the R programming language in the script \code{example.R}. This caption is wrapped at the right width, and the height is being compensated.}
    \label{fig:xxxa}
  \end{subfigure}
  \hfill
  \begin{subfigure}[t]{0.45\textwidth}
    \includegraphics[width=\textwidth]{./figs/fig_02.png}
    \caption{This figure was generated in the Python programming language. The script \code{example.py} creates this figure and the additional table.}
    \label{fig:xxxb}
  \end{subfigure}
  
  \captionsetup{margin=10pt,font=small,labelfont=bf,width=.8\textwidth}

  \caption[Short caption 2]{This is the main caption and it is below the figures. Both figures were automatically created in scripts. If we want to change the figure, we change the script only. \textit{Source:} own calculations}\label{fig:xxx}
\end{figure}


\begin{table}[hbt]
  \centering

  \captionsetup{margin=10pt,font=small,labelfont=bf,width=.8\textwidth}

  \caption[Pandas table]{This is another table generated by the Python script \code{example.py}. This table looks a little bit different, but it's acceptable.}
  \label{tab:pandasdf}

  \vspace*{2ex}
  \footnotesize

  \begin{tabular}{lr}
\toprule
Class & Values \\
\midrule
j & 0.953570 \\
M & 0.183956 \\
X & 1.243109 \\
I & -1.032789 \\
N & 0.443236 \\
K & -1.602915 \\
l & -1.273745 \\
l & 2.209001 \\
r & 0.190158 \\
R & 0.873841 \\
\bottomrule
\end{tabular}

  
\end{table}


% --- chapter ---------------------------------------------------------
\clearpage
\section{Bibliography}

\begin{wrapfigure}{r}{.5\textwidth}
\centering

\includegraphics[width=.4\textwidth]{./figs/fig_02.png}

\captionsetup{margin=10pt,font=small,labelfont=bf,width=.42\textwidth}

  \caption[Short caption 2]{This is how one can wrap a text around a figure. \textit{Source:} own calculations}\label{fig:yyy}


\end{wrapfigure}

The content for the bibliography is in a different file named \verb+refs.bib+. You can change the name but then you have to change the information in this file from \verb+\bibliography{refs}+ to \verb+\bibliography{new-name}+ where \verb+new-name+ is the name of your file. The file \verb+refs.bib+ contains some examples for books and papers.

The process of citation is simple. The command  \verb+\textcite{garland2010}+ gives this \textcite{garland2010} and puts all information into the bibliography section at the end. Everything is sorted and formatted, so you don't have to worry about this. An example of a paper with many authors is \cite{benaim2003} or \cite{osborne1998}. We can cite online resources \cite{bbb} or \cite{cnn}. We can use the following citation \parencite{benaim2003} or like this \footcite{benaim2003} or like this \footfullcite{cnn}.

% --- appendices ------------------------------------------------------
\appendix

% ---------------------------------------------------------------------
\clearpage
\section{Appendix: Some important stuff}

This is an appendix. This is the place to put it if you have some additional figures, tables, or a code. The really long tables or really wide tables should be placed in additional files e.g., XLSX. 

\lstinputlisting[firstline=51,lastline=60,basicstyle=\ttfamily \footnotesize \color{black}]{../code_r/example.R}

Below, there is a fragment of the \code{example.py} script that creates a Pandas tabel and exports to \LaTeX{}.

\lstinputlisting[firstline=16,lastline=29,basicstyle=\ttfamily \footnotesize \color{black}]{../code_python/example.py}

% --- bibliography ----------------------------------------------------
\clearpage
\printbibliography[heading=subbibliography,nottype=online,title={References}]
\printbibliography[heading=subbibliography,type=online,title={Online references}]


% --- abstract --------------------------------------------------------
\clearpage
\addcontentsline{toc}{section}{List of tables}
\listoftables

% --- abstract --------------------------------------------------------
\clearpage
\addcontentsline{toc}{section}{List of figures}
\listoffigures



% --- abstract --------------------------------------------------------
\clearpage
\addcontentsline{toc}{section}{Streszczenie}
\section*{Streszczenie}

Tutaj zamieszczają Państwo streszczenie pracy. Streszczenie powinno być długości około pół strony.


\end{document}


%%% Local Variables:
%%% mode: latex
%%% TeX-master: t
%%% End:
